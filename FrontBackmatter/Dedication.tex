%*******************************************************
% Dedication
%*******************************************************

\newpage\thispagestyle{empty}

\textit{To my parents.}


\bigskip
\bigskip
\bigskip

\noindent A new sense of the notion of information has been constructed around the photographic image.
The photograph is a thin slice of space as well as time.
In a world ruled by photographic images, all borders (``framing'') seem arbitrary\ldots
    %Anything can be separated, can be made discontinuous, from anything else: all that is necessary is to frame the subject differently. (Conversely, anything can be made adjacent to anything else.) Photography reinforces a nominalist view of social reality as consisting of small units of an apparently infinite number--as the number of photographs that could be taken of anything is unlimited. Through photographs, the world becomes a series of unrelated, freestanding particles; and history, past and present, a set of anecdotes and faits divers...  \\
    %The camera makes reality atomic, manageable, and opaque. It is a view of the world which denies interconnectedness, continuity, but which confers on each moment the character of a mystery. Any photograph has multiple meanings; indeed, to see something in the form of a photograph is to encounter a potential object of fascination.
The ultimate wisdom of the photographic image is to say:
    \begin{quote}
      ``There is the surface. Now think--or rather feel, intuit--what is beyond it, what the reality must be like if it looks this way.''
      \end{quote}
\noindent  --- Susan Sontag, \emph{On Photography}


\vfill%
% \begin{center}
%     To my parents. \\
% \end{center}
