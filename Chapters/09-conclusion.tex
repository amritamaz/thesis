\chapter{Conclusion}
\label{ch:concl}
\markboth{Conclusion}{Conclusion}

In this thesis, I described a series of systems for applying perceptual information to video capture, processing and storage systems.
The computer systems and architectures targeted by the work in this dissertation span software and hardware design, and leverage application insights and codesign principles to fully leverage perceptual optimizations.
I described hardware-software codesign work for mobile vision on energy-harvesting cameras, real-time virtual reality video synthesis, and energy-efficient mobile video playback.
For cloud-scale video systems, I demonstrated how saliency can be used to improve video storage and processing.

Video data continues to grow with increased video capture and consumption trends. Leveraging perceptual cues can help manage this data.
By abstracting per-pixel information into a pereceptual layer, we can optimize computer system performance based on an image's visual structure, semantics, or saliency.
My thesis presents several examples of applying perceptual cues to modern visual computing tasks to achieve faster or real-time performance, improved energy efficiency, or smaller data bandwidth.

Looking forward, visual media will undoubtedly continue to drive use of computational reesources.
Vision and computational photography algorithms will continue to improve.
Demand for richer displays and cameras will continue to grow.
Compute infrastructure will become increasingly more specialized.
While traditional systems concerns like performance and energy efficiency will drive new system and architecture designs, systems will also make use of more perceptual information to deliver improved performance.
Moreover, computer systems will need to optimize not only for \emph{intrinsic} perceptual cues in visual media, but also perceptual characterisitics of the end-application.
Visual computing workloads for human viewing consumption, like HFBS, Vignette and TVA, highlight different optimization criteria from workloads meant for machine consumption, like the near-sensor processor.
As application-specific systems bifurcate into optimized workloads for algorithms or end-users, this thesis on perceptually-optimized visual computing workloads and architectures can serve as a launch point for future visual computing architectures.
