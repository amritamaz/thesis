\chapter{Conclusion}
\label{ch:concl}
\markboth{Conclusion}{Conclusion}

This document presents a research vision for applying perceptual information to optimize video capture, processing and storage systems.
The computer systems and architectures targeted by the work in this document span software and hardware design, and leverage application insights and codesign principles to fully leverage perceptual optimizations.
In this thesis proposal, I presented several projects, highlighting the use of different perceptual cues in the optimization of visual computing systems. I described an accomplished work of hardware-software codesign for mobile vision on energy-harvesting cameras and real-time virtual reality video synthesis. I also proposed a hardware-software codesign project on perceptual color rendering for mobile video displays.
For cloud-scale video systems, I showed accomplished work in using saliency for video storage, and proposed new work on applying AI-friendly saliency models for video analytics.

Video data continues to grow with increased video capture and consumption trends, but leveraging perceptual cues can help manage this data.
My thesis describes several examples of applying perceptual cues to modern visual computing tasks to achieve faster or real-time performance, improved energy efficiency, or smaller data bandwidth.
As vision and computational photography algorithms continue to improve, display and camera production continues to increase, and compute infrastructure becomes increasingly specialized, I expect that future efforts will continue to build on these strategies to deliver immersive, real-time visual experiences.
