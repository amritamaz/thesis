\chapter{Conclusion}
\label{ch:concl}
\markboth{Conclusion}{Conclusion}

In this thesis, I described a series of systems for applying perceptual information to video capture, processing and storage systems.
The computer systems and architectures targeted by the work in this dissertation span software and hardware design, and leverage application insights and codesign principles to fully leverage perceptual optimizations.
I described an accomplished work of hardware-software codesign for mobile vision on energy-harvesting cameras, real-time virtual reality video synthesis, and energy-efficient mobile video playback.
For cloud-scale video systems, I demonstrated how saliency can be used to improve video storage and processing.

Video data continues to grow with increased video capture and consumption trends, but leveraging perceptual cues can help manage this data.
My thesis presents several examples of applying perceptual cues to modern visual computing tasks to achieve faster or real-time performance, improved energy efficiency, or smaller data bandwidth.
As vision and computational photography algorithms continue to improve, display and camera production continues to increase, and compute infrastructure becomes increasingly specialized, I expect that future efforts will continue to build on these strategies to deliver immersive, real-time visual experiences.
