\section{Summary}
This section proposes integrating perceptual compression techniques with cloud video storage infrastructure to improve storage capacity and video bitrates while maintaining perceptual quality.
\name combines automatic generation of perceptual information with a video transcoding pipeline to enable large-scale perceptual compression with minimal data overhead.
Our storage system supports a feedback loop of perceptual compression, including updates as an application gathers data from sources such as eye trackers.
Our offline compression techniques deliver storage savings of up to 95\%, and user trials confirm no perceptual quality loss for \name videos 50-75\% smaller in size.

Our limit study of video transcoding shows that significant opportunity for compression can be uncovered by the inclusion of perceptual information.
% \todo{some statistics from limit study here}
\name describes a feasible implementation for managing perceptual information and leveraging it to reduce video data size as more perceptual data is collected.
\name's design complements the contributions of existing large-scale cloud video storage and processing systems.
Video systems can use \name to further improve storage capacity or in anticipation of video workloads that produce perceptual information.
As users adopt new technologies to capture more perceptual cues, \name's techniques can be extended to utilize these perceptual cues as well.
As VR video consumption and new perceptual markers---such as eye trackers in VR headsets---grow in popularity, \name's techniques will be critical in integrating perceptual compression at large scale for higher quality, lower bitrate video. %reduce storage and transmission load on their video streaming and storage infrastructure
