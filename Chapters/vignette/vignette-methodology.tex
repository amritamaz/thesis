%!TEX root = ../paper.tex

\subsection{Methodology}
\label{sec:vignette-methodology}

We implement \name by extending \lightdb~\cite{lightdb}, a database management system for VR videos.
\lightdb{} lets developers declaratively express queries over large-scale video and uses a rule-based optimizer to maximize performance.
Developers can easily express \hevc-based saliency encoding in \lightdb's query language by combining its \texttt{Encode}, \texttt{Partition}, and \texttt{Subquery} operators:
% \vspace{-0.5em}
\begin{lstlisting}[style=VRQL]
Decode("rtp://...")
  >> Partition(Time, 1, Theta, $\pi$ / rows, Phi, $2\pi$ / cols)
  >> Subquery([](auto& partition) {
        return Encode(partition, $saliency\_mapping$(partition) })
  >> Store("output");
\end{lstlisting}
% \vspace{-0.5em}

\noindent In this example, \texttt{Partition} divides the input video into tiles, \texttt{Encode} transcodes each tile with the corresponding \texttt{saliency\_mapping} value as an argument, and  \texttt{Subquery} executes the given operation over all the partitioned tiles.
We also wrote our object recognition queries for \ref{subsec:qos} in \lightdb to simulate video analytics workloads.
To generate saliency maps, we used MLNet~\cite{mlnet2016} with publicly-available weights trained on the SALICON~\cite{huang2015salicon}, which achieves 94\% accuracy on the MIT300 saliency benchmark.

\noindent\textbf{Baseline: } We compare \name against the HEVC encoding implementations included with \texttt{FFmpeg}.
We configure \texttt{FFmpeg} with support for NVENCODE~\cite{nvenc} GPU-based encoding of \hevc video, as it is supported by large-scale video services and devices~\cite{de2016large}.
We also implement \nameCompress on top of \texttt{FFmpeg} version \texttt{n4.1-dev}, and use the GPU-based NVENC HEVC encoder for tiled encoding.
Unless otherwise specified, we target a constrained bitrate using maximum bitrate mode (VBV) to rapidly generate results.

We performed all experiments on a single-node server running Ubuntu 16.04 and containing an Intel i7-6800K processor (3.4 Ghz, 6 cores, 15 MB cache), 32 GB DDR4 RAM at 2133 MHz, a 256 GB SSD drive (ext4 file system), and a Nvidia P5000 GPU with two discrete NVENCODE chipsets.

\noindent \textbf{Video Workload Datasets}: We use a collection of video datasets, listed in Table~\ref{table:benchmarks}, to evaluate the impact of our techniques across different classes of video.
Standard video formats and emerging VR formats comprise our evaluation datasets.
The former include representative workloads from Netflix~\cite{netflix2016data} and YouTube~\cite{vbench}.
The VR and emerging video datasets highlight demands of ultra high-definition (UHD) formats such as 360$^\circ$ video~\cite{saliency-map} and the Blender stereoscopic and UHD open source films~\cite{blender}.
To construct a representative sampling of Blender video segments, we partitioned the movies in the Blender dataset (``Elephants Dream'', ``Big Buck Bunny'', ``Sintel'', and ``Tears of Steel'') into 12-second segments, and selected five segments that covered the range of entropy rates present in each film.

\benchmarkInformationFigure

In this collection of datasets, we found that the \texttt{vbench} ``desktop'' video, a 5-second computer screencast recording, responded poorly during all compression evaluations because of its low entropy and content style, so we excluded it from our evaluation results.
We discuss this style of video in relation to \name further in \ref{sec:related}.
We also replaced Netflix's single ``Big Buck Bunny'' video segment with the same video content from Blender's stereoscopic, 4K, 60 frames-per-second version of the video.

\noindent \textbf{Quantitative Quality Metrics}:
We measured video encoding quality using two quality metrics, peak signal-to-noise ratio (PSNR) and eye-weighted PSNR (EWPSNR).
% While manual inspection or user studies are the ideal metric for user satisfaction, they are unsustainable to deploy for large-scale video encoding systems.
% Instead, researchers use quantitative metrics to approximate quality.
PSNR reports the ratio of maximum to actual error per-pixel, in decibels (dB), by computing the per-pixel mean squared error and comparing it to the maximum per-pixel error.
% PSNR is typically computed for each of the three channels in the YCbCr video color space and averaged to produce a single average PSNR value.
PSNR is popular for video encoding research, but researchers acknowledge that it fails to capture some obvious perceptual artifacts~\cite{netflix2016data}.
Acceptable PSNR values fall between 30 and 50 dB, with values above 50 dB considered to be lossless~\cite{vbench}.
For saliency prediction evaluations, researchers developed eye-weighted PSNR to more accurately represent human perception~\cite{li2011visual}.
EWPSNR prioritizes errors perceived by the human visual system rather than evaluating PSNR uniformly across a video frame.
We computed EWPSNR using the per-video saliency maps described in~\ref{sec:tiled-compression} as ground truth.
% This metric uses the same core metric as PSNR; however, it accommodates more perceptual qualities but still fails to capture perceptual artifacts.

% Using the VMAF phone model is more generous to distortions from comparing to an already-encoded video.

% We measured quality using all three metrics: PSNR, EWPSNR, and VMAF.
% When possible, we report all results.
% For simplicity, we considered only PSNR when generating our per-video bitrate ladders to best mirror reported practices across the largest number of video services.
