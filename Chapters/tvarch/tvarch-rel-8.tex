\section{Related Work}

To our knowledge, this is the first work to evaluate the impact of perceptual compression on conventional video decode accelerators.
Our work combines ideas from heterogeneous parallel accelerator design, mobile visual computing, and foveated compression; we discuss each in turn.

\textbf{Parallel and heterogeneous accelerator design. } \nameArch's design borrows heavily from CPU-based single-ISA heterogeneous system proposals~\cite{Kumar2003micro}.
The design of \nameArch, using a parallel configuration of big/little cores, is inspired by ARM's big.LITTLE architecture and other similar design proposals~\cite{armBigLittle}.
\nameArch's hardware scheduler design, in particular, resembles recent heterogeneous architecture designs for other workloads~\cite{pie2012isca, 6522303}.

\textbf{Mobile visual computing accelerators. }Architecture requirements for mobile visual computing have grown more stringent as algorithms and compute complexity have increased at the cost of power and energy efficiency.
Recent work has improved energy-efficiency across the mobile rendering pipeline by investigating image processing \cite{pu2017taco, qadeer}, VR video rendering~\cite{evr19isca}, and mobile displays \cite{zhang2017race, crayon}, but few works include the decode unit in optimizations.
EVR~\cite{evr19isca} proposes augmenting the mobile SoC with processing to perform the perspective transform on decoded VR video frames.
Multiple recent proposals target optimizing rendering for semantic objects in VR \cite{xie2019isca,leng2018semantic}.
Future work could examine a hardware acceleration fabric that combines these hardware optimizations into a single chip, or fully redesign the pipeline to best optimize data transfer across the VR video workload.

% Prior studies examine mobile energy consumption, visual computing workloads, and VR hardware, this is the first work to examine the growing trend of perceptually compressed video content on conventional video decoder ASICs.
\textbf{Heterogeneous video processing architectures. }The most closely related work evaluates the impact of parallel decoder cores on a \avc chip design~\cite{multicore-avc}.
In that work, Finchelstein~\etal evaluate using multiple cores and the resulting power and area of a parallel video decoder.
Their design, however, predates the trend of tiled video and they do not design for this type of parallelism.
Tikekar \etal contribute a related parallel video decoder design using embedded DRAM to reduce data movement for VR devices~\cite{tikekar18ijssc}.
Boroumand \etal evaluate executing video decode hardware with processing-in-memory (PIM) to reduce energy from data movement~\cite{google2018asplos}; combining \nameArch's parallelism with PIM integration~\cite{8675230} or codesigned caching~\cite{zhang2017race} could further improve performance.

\textbf{Virtual reality video compression. }Foveated compression is well-studied in video coding and perceptual science research, but it has grown even more popular with the emergence of consumer VR headsets \cite{geisler1998real, albert}.
Google's foveation pipeline uses multi-resolution compression~\cite{google-foveation}.
Furion leverages parallel cores to achieve fast tile-based perceptual rendering for games, but their decode is implemented in software and limited to four available mobile CPU cores~\cite{furion2019}.
Tile-based streaming has become a popular method to deliver video content adapted to a VR viewer's head location~\cite{liu2018mobisys}.
Facebook recently described an AI-based decoder for sparse foveated video, requiring machine learning infrastructure to decode frames~\cite{kaplanyan2019deepfovea}.
Coupling foveated sensors and compression has been evaluated recently to optimize video capture and encoding~\cite{8493507}; this work targets the inverse problem of decoding foveated video.
\nameArch's proposal to use additional video metadata to improve decode performance dovetails with recent VR video proposals to use metadata communicating application-specific parameters \cite{vignette,bhaynes_vldb18}.
