\section{Summary}
Emerging techniques for perceptually compressing videos provide network bandwidth benefits, but video decoder architectures have not innovated alongside them.
To adapt VR video decoder architectures with energy efficiency in mind, this paper proposed \nameArch, a new parallel accelerator optimized for VR video decompression.
\nameArch's design uses heterogeneous parallel cores to minimize decode latency and distribute decode effort efficiently, reducing energy by an average of 12--22\% per frame.
Moreover, \nameArch can efficiently decode perceptually compressed videos to achieve 4.5$\times$ better energy efficiency over traditionally compressed video.
Our tool, \nameArchprof, helps VR developers characterize the energy consumption of emerging perceptually-compressed video techniques on video decoder hardware.
\nameArchprof can help VR software developers and ASIC designers better tune their workloads and optimize energy consumption as the VR software and hardware ecosystem continues to evolve.
As performance demands for video encoding and processing techniques grow more stringent, \nameArch and \nameArchprof help to close the gap for optimizing the energy consumption of future video workloads.
