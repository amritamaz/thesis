%************************************************
\chapter{Introduction}\label{ch:introduction}
%************************************************
Visual media, namely photos and videos, is a fundamental driver of computer systems applications.
Compressed video data, for instance, constitutes 70\% of Internet traffic and is stored in hundreds of combinations of codecs, qualities, and bitrates~\cite{netflix2015encoding, cisco2017zettabyte}.
These videos are produced by increasingly inexpensive and low-power mobile cameras, deployed at scale in everyday consumer devices like mobile phones and home sensors, as well as in city infrastructure surveillance systems, modern cars, and wildlife refuges.
New domains of video production and viewing---e.g., panoramic (\threesixty), stereoscopic, and light field video for virtual reality (VR)---further compound the production of visual media.
Furthermore, improvements in display technologies and computer vision algorithms motivate not just the extensive use of cameras and video capture, but also consumption of increased volume of video content.
We are not just producing \emph{more} photos and video, but also at higher resolutions, faster frame rates, and with larger dynamic ranges of color, and distributing them to larger groups of users.

As these video and graphics applications become richer and more immersive, however, they push the limits of already-strained computer systems.
For decades, computer architecture has leveraged the easy gains of Dennard scaling to provide exponential performance with little added complexity to general purpose computer processing units (CPUs)~\cite{esmaeilzadeh2011dark}.
As CPU design has reached its physical limits, computer architects have looked to designing custom hardware to achieve the increased compute and memory specifications required for modern computing workloads~\cite{tpu, q100, qadeer}.
By tailoring the design of compute infrastructure to meet the needs of specialized visual computing workloads, these systems can improve speed or power consumption, leveraging hardware like graphics processing units (GPUs), general purpose GPUs (GPGPUs), field programmable gate arrays (FPGAs), and custom application specific integrated circuits (ASICs).

\section{Challenges for Visual Computing Systems}
Visual computing applications are a natural target for specialized custom hardware: the two-dimensional image representation is easily expressed using array and buffer primitives; many image processing filters are simple convolutional kernels easy to broadcast in parallel.
But the cost of data movement (both within a single hardware design and across networks to different chips, modules, or computers in a datacenter) for these large volumes of visual media for VR and video processing workloads far exceeds the benefits of tightly optimized parallel algorithms.
Managing this data requires more holistic design approaches beyond parallel compute patterns and stream processing.

Fundamentally, images and video data use pixel-level information (RGB color at XY coordinates over time) to communicate higher-level information (motion, objects, visual importance, scene meanings).
However, past work in specialized image and video hardware accelerators~\cite{hauswald2014hybrid, convolution_engine, adams2010frankencamera} only optimizes hardware at the pixel level, processing every pixel uniformly and measuring tradeoffs in designs by evaluating pixel-level metrics.

While architectural challenges have motivated system designers to push performance limits by codesigning custom systems for visual media applications, this matrix representation of visual media has remained stagnant.
On the other hand, many visual computing applications produce results at a frame or video segment granularity, as opposed to pixel-level results.
For these applications, integrating this higher-level intent in the hardware-software codesign process introduces new opportunities for improving performance as well as new challenges in system design.

\section{Thesis Overview}
This dissertation posits that \emph{codesigning video-focused systems with perceptual information} improves performance and energy efficiency while maintaining application quality of service.
To support this, I employed techniques from the domains of hardware-software codesign and machine learning for systems optimization.
Combining these computer systems and architecture methods with vision and graphics research on extracting meaningful information from 2D visual media, this dissertation demonstrates how to improve the performance of constrained video processing systems while maintaining high quality of experience.

The visual computing systems I considered operate on images and video at various stages of the visual computing pipeline: during the \emph{capture}, \emph{processing}, or \emph{storage} phases, or some combination therein.
In existing work, hardware-software codesign alone has been shown to improve performance in these domains (Chapter~\ref{ch:related}).
My thesis builds upon this prior work to show how the use of perceptual optimizations can expand the design space and help identify new accelerator designs.
The resulting systems demonstrate how to use perceptual cues to improve the power efficiency, latency, and storage capacity of video management systems, and they highlight how treating visual media at a richer granularity than matrix arrays of pixels uncovers new opportunities for optimizing system performance.

I define \textit{perceptual optimizations} as a class of optimizations that improve system performance by targeting a quality metric or perceptual measure corresponding to higher-level meaning in the image or video, as opposed to per-pixel quality.
This dissertation considers three classes of perceptual optimizations: \textit{structure}, \textit{semantics}, and \textit{saliency}.
To explore these perceptual cues, I used hardware-software codesign and machine learning for systems techniques to optimize computer systems at different points in the video processing pipeline.

\subsection{Structure}
Structural perceptual cues encode detailed information about the structure of the  scene in the visual frame.
For instance, simple convolutional kernels can be used to extract \emph{edges} from the scene, an elementary form of structure to draw boundaries~\cite{FFLS2008}.
More complex computational photography algorithms can generate \emph{motion}, \emph{depth}, and \emph{lighting} from a series of images or video~\cite{optical-flow, googlejump, Barron2015A}.
Making use of these perceptual cues allows for applications to generate new viewpoints or lighting conditions or understand the distance between pixels in 3D space.
Structural perceptual cues can be captured from complex sensors like depth or IR sensors, or synthetically generated through image processing or more complex computational photography and machine learning algorithms.

Chapter~\ref{ch:hfbs} makes use of structural cues to rapidly process virtual reality video from a complex camera rig.
In that work, I used structural information about color and depth in captured video to design an algorithm for fast, accurate 3D-\threesixty video generation (Chapter~\ref{ch:hfbs}).
In particular, I formulated a new data structure for processing video captured from VR camera rigs into visually pleasing 3D-\threesixty video, and design a hardware accelerator around this data structure for real-time VR video generation.
I proposed a new algorithm, the hardware-friendly bilateral solver, that enables faster runtimes than existing algorithms of similar quality.
I also designed an FPGA-based hardware accelerator that utilizes reduced-precision computation and the parallelism inherent in our algorithm to achieve further speedups over our CPU and GPU implementations while consuming an order of magnitude less power.
Combining the FPGA design's power efficiency with our structure-aware algorithm enabled practical, real-time VR video processing at the camera rig or in the cloud.

\subsection{Semantics}
Semantic perceptual cues generate information about the \emph{meaning} of the scene encoded in the visual frame.
For example, the number of objects detected by an object detection algorithm~\cite{vj_journal} is a semantic cue.
A more expressive semantic cue would use an object recognition algorithm to identify each object and map the label to a bounding box or location~\cite{redmon2017yolo}.
These semantic cues provide value because they give insight into what the user perceives in the frame, beyond the colors at each pixel location.
Semantic perceptual cues can be generated by crowdsourced user labeling, or, as in more recent systems for machine learning works, generated by use of computer vision algorithms on large batches of visual media~\cite{poms2018scanner}.

In Chapter~\ref{ch:near-sensor}, I evaluated the use of semantic information in the design of hardware ASICs for an energy-harvesting camera system.
That work considered not only the use of semantic cues, but also how to balance computational complexity with the accuracy and specificity of algorithms generating semantic information.
I composed two accelerators---a simple machine learning algorithm and a more computationally intense one---to conduct semantic filtering for improved energy efficiency.
I proposed an organization of the hardware designs where one specialized accelerator first filters for one granularity of semantic information and then offloads processing to a more complex neural network to generate more detailed semantic data.
To implement the desings, I employed several energy savings techniques such as reduced-precision functional units, low-voltage logic and memory, hardware specialization, and take full advantage of the accuracy $\times$ resource usage trade-off offered by the ML algorithms.
The results showed \textasciitilde100$\times$ improvement in performance and energy efficiency, enabling applications that would not be possible without a semantics-aware hardware accelerator.

\subsection{Saliency}
The perceptual cue of saliency encodes a spatial measure of visual importance.
Within a single frame of video or an image, an algorithm could generate per-pixel saliency data to capture another dimension of information about the frame.
Compared to semantic cues, which capture information about \emph{what} a space of pixels means, saliency cues only encode the extent to which the space of pixels is relevant or important.

Traditionally, saliency in computer applications referred to human visual saliency, or what humans visually perceived to be salient~\cite{itti1998model}.
Human visual saliency data can be collected using user trials with eye trackers or mouse movements, or generated automatically using computer vision algorithms.

In this dissertation, I used saliency to explore the design of cloud video processing systems and mobile video playback hardware.
In Chapter~\ref{ch:vignette}, I used human visual saliency to optimize video compression and show the design of a storage system built codesigned with an understanding of this perceptual cue.
The compression technique used a neural network to predict saliency information used during transcoding, and the storage manager integrated perceptual information into the video storage system with little overhead.
The results demonstrated the benefit of embedding information about the human visual system into the architecture of cloud video storage systems.

In Chapter~\ref{ch:tvarch}, I leveraged saliency cues in the form of foveated video content to design new mobile video decode hardware with increased energy efficiency.
By observing that VR video combines small, high-resolution regions with large, low-resolution areas, I devised a new parallel, energy-efficient decode architecture.
This architecture schedules work across parallel heterogeneous video decode cores in order to balance the compression of both high- and low-resolution video content.
The accompanying profiling tool I designed helps VR developers explore and characterize the energy-consumption of video workloads under various perceptual compression schemes.
These works demonstrated how automatically-generated saliency information from machine learning algorithms can optimize both large-scale video processing pipelines and power-constrained mobile video playback hardware.

\section{Organization}
This chapter serves as the introductory framing for the remainder of the thesis.
In the following chapter, I more deeply contextualize my work within the areas of specialized computer architectures and visual computing systems, and highlight prior work using the perceptual cues I consider.

Chapter~\ref{ch:near-sensor} describes my work codesigning computer vision hardware for energy-harvesting cameras, balancing tradeoffs of accuracy and energy efficiency through the use of semantic perceptual information. Chapter~\ref{ch:hfbs} highlights another hardware-software codesign solution, applied to a 16-camera virtual reality video pipeline, using perceptual optimizations based on the structural content of distance and depth to improve the algorithm's accuracy and speed.

Chapters~\ref{ch:vignette} and~\ref{ch:tvarch} describe saliency-optimized systems. Chapter~\ref{ch:vignette} presents Vignette, a storage system and compression algorithm designed to support applying human visual saliency information to traditional video compression for improved storage and streaming bandwidth.
Chapter~\ref{ch:tvarch} describes how new mobile device hardware can adapt for a future of saliency-optimized videos to improve performance and energy consumption.

Finally, Chapter~\ref{ch:concl} concludes with notes for future perceptually-optimized visual computing systems.

\section{Previously Published Material}

This dissertation includes work published as conference papers:

\begin{itemize}
  \item Chapter~\ref{ch:near-sensor}: \fullcite{amazumdar_iiswc17}
  \item Chapter~\ref{ch:hfbs}: \fullcite{amazumdar_hpg17}
  \item Chapter~\ref{ch:vignette}: \fullcite{vignette}
\end{itemize}


%*****************************************
%*****************************************
%*****************************************
%*****************************************
%*****************************************
